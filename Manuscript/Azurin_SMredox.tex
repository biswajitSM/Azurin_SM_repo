
\documentclass[journal=jacsat,manuscript=article]{achemso}

\usepackage[version=3]{mhchem} % Formula subscripts using \ce{}
\usepackage[T1]{fontenc}       % Use modern font encodings

\newcommand*\mycommand[1]{\texttt{\emph{#1}}}

\author{Andrew N. Other}
\altaffiliation{A shared footnote}
\author{Fred T. Secondauthor}
\altaffiliation{Current address: Some other place, Othert\"own,
Germany}
\author{I. Ken Groupleader}
\altaffiliation{A shared footnote}
\email{i.k.groupleader@unknown.uu}
\phone{+123 (0)123 4445556}
\fax{+123 (0)123 4445557}
\affiliation[Unknown University]
{Department of Chemistry, Unknown University, Unknown Town}
\alsoaffiliation[Second University]
{Department of Chemistry, Second University, Nearby Town}
\author{Susanne K. Laborator}
\email{s.k.laborator@bigpharma.co}
\affiliation[BigPharma]
{Lead Discovery, BigPharma, Big Town, USA}
\author{Kay T. Finally}
\affiliation[Unknown University]
{Department of Chemistry, Unknown University, Unknown Town}
\alsoaffiliation[Second University]
{Department of Chemistry, Second University, Nearby Town}

\title[An \textsf{achemso} demo]
  {A demonstration of the \textsf{achemso} \LaTeX\
   class\footnote{A footnote for the title}}

%%%%%%%%%%%%%%%%%%%%%%%%%%%%%%%%%%%%%%%%%%%%%%%%%%%%%%%%%%%%%%%%%%%%%
%% Some journals require a list of abbreviations or keywords to be
%% supplied. These should be set up here, and will be printed after
%% the title and author information, if needed.
%%%%%%%%%%%%%%%%%%%%%%%%%%%%%%%%%%%%%%%%%%%%%%%%%%%%%%%%%%%%%%%%%%%%%
\abbreviations{IR,NMR,UV}
\keywords{American Chemical Society, \LaTeX}

%%%%%%%%%%%%%%%%%%%%%%%%%%%%%%%%%%%%%%%%%%%%%%%%%%%%%%%%%%%%%%%%%%%%%
%% The manuscript does not need to include \maketitle, which is
%% executed automatically.
%%%%%%%%%%%%%%%%%%%%%%%%%%%%%%%%%%%%%%%%%%%%%%%%%%%%%%%%%%%%%%%%%%%%%
\begin{document}

\begin{tocentry}

\end{tocentry}

\begin{abstract}

\end{abstract}

%%%%%%%%%%%%%%%%%%%%%%%%%%%%%%%%%%%%%%%%%%%%%%%%%%%%%%%%%%%%%%%%%%%%%
%% Start the main part of the manuscript here.
%%%%%%%%%%%%%%%%%%%%%%%%%%%%%%%%%%%%%%%%%%%%%%%%%%%%%%%%%%%%%%%%%%%%%
\section{Introduction}


\section{Results and discussion}

\subsection{Outline}


%\begin{scheme}
%
%\end{scheme}

%\begin{figure}
%
%\end{figure}


%\begin{equation}
% 
%\end{equation}


\section{Experimental}

%\begin{table}
%
%\end{table}


%\begin{acknowledgement}
%
%\end{acknowledgement}

%%%%%%%%%%%%%%%%%%%%%%%%%%%%%%%%%%%%%%%%%%%%%%%%%%%%%%%%%%%%%%%%%%%%%
%% The same is true for Supporting Information, which should use the
%% suppinfo environment.
%%%%%%%%%%%%%%%%%%%%%%%%%%%%%%%%%%%%%%%%%%%%%%%%%%%%%%%%%%%%%%%%%%%%%
\begin{suppinfo}

%\begin{itemize}
%  \item Filename: brief description
%  \item Filename: brief description
%\end{itemize}

\end{suppinfo}

%%%%%%%%%%%%%%%%%%%%%%%%%%%%%%%%%%%%%%%%%%%%%%%%%%%%%%%%%%%%%%%%%%%%%
%% The appropriate \bibliography command should be placed here.
%% Notice that the class file automatically sets \bibliographystyle
%% and also names the section correctly.
%%%%%%%%%%%%%%%%%%%%%%%%%%%%%%%%%%%%%%%%%%%%%%%%%%%%%%%%%%%%%%%%%%%%%
\bibliography{achemso-demo}

\end{document}
